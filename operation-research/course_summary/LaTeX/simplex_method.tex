\section{单纯形法}
\subsection{单纯形法}
单纯形法 (simplex method) 的每一步都在引入一个非基变量取代某一基变量,找出目标函数值更优的另一基本可行解,逐渐调整到最优解。 \\
有以下步骤:
\begin{enumerate}
    \item 把线性规划问题的约束方程组表达成典范型方程组,找出基本可行解作为初始基可行解。
    \item 若基本可行解不存在,即约束条件有矛盾,则问题无解。
    \item 若基本可行解存在,从初始基可行解作为起点,根据最优性条件和可行性条件,引入非基变量取代某一基变量,找出目标函数值更优的另一基本可行解。
    \item 按步骤3进行迭代,直到对应检验数满足最优性条件(这时目标函数值不能再改善),即得到问题的最优解。
    \item 若迭代过程中发现问题的目标函数值无界,则终止迭代。
\end{enumerate}
假设有以下问题:
$$
\max \quad 3x_1 + 2x_2
$$
$$
\text{s.t.} 
\begin{cases}
    2x_1 + x_2 \le 12 \\
    x_1 + 2x_2 \le 9 \\
    x_1, x_2 \ge 0
\end{cases}
$$
加入松弛变量,将原问题变为 
$$
\max \quad 3x_1 + 2x_2
$$
$$
\text{s.t.} 
\begin{cases}
    2x_1 + x_2 + x_3 = 12 \\
    x_1 + 2x_2 + x_4 = 9 \\
    x_1, x_2, x_3, x_4 \ge 0
\end{cases}
$$
此时,有初始可行解:$x = \begin{bmatrix} 0 & 0 & 12 & 9 \end{bmatrix}^T$,当前目标函数值为 $z = 0$。将各个变量用非基变量进行表示,有
$$
\begin{cases}
    x_3 = 12 - 2x_1 - x_2 \\ 
    x_4 = 9 - x_1 - 2x_2 \\ 
    z = 3x_1 + 2x_2
\end{cases}
$$
$x_1$ 的系数比较大(这个系数称为\textbf{检验数}),将 $x_1$ 变成基变量。根据 $x_1$ 与 $x_3$ 和 $x_4$ 之间的表达式不难看出,当 $x_1 = 6$ 时,$x_3$ 最先变成 0,将它从基变量中去除。
此时,可行解变为 $x = \begin{bmatrix}6 & 0 & 0 & 3\end{bmatrix}^T$,当前目标函数值为 $z = 18$。将各个变量用非基变量进行表示,有
$$
\begin{cases}
x_1 = 6 - x_2/2 - x_3/2 \\ 
x_4 = 3 - 3x_2/2 + x_3/2 \\ 
z = 18 + x_2/2 - 3x_3/2
\end{cases}
$$
从系数可以看出,只有增加 $x_2$ 才能增大目标函数值,将 $x_2$ 变成基变量。从 $x_2$ 与 $x_1$ 和 $x_4$ 的关系式中也不难发现,当 $x_2 = 2$ 时,$x_4$ 最先变成 0,将它从基变量中去除。
此时,可行解变为 $x = \begin{bmatrix} 5 & 2 & 0 & 0 \end{bmatrix}^T$,当前目标函数值为 $z = 19$。将各个变量用非基变量进行表示,有 
$$
\begin{cases}
x_1 = 5 - 2x_3/3 + x_4/3 \\ 
x_2 = 2 + x_3/3 - 2x_4/3 \\ 
z = 19 - 4x_3/3 - x_4/3
\end{cases}
$$
可以发现,$x_3$ 和 $x_4$ 与 $z$ 相关的系数都是负值,此时无论把哪个变量加入基变量,都只能让目标函数值变小了。此时得到了线性规划问题的最优解:$x = \begin{bmatrix} 5 & 2 & 0 & 0 \end{bmatrix}^T$,目标函数值为 $z = 19$。

\subsection{单纯形表}
单纯形表 (simplex tableau) 用矩阵的形式,将单纯形法中各个变量用非基变量进行表示。
假设有以下线性规划问题:
$$
\max \quad z = c^Tx
$$
$$
\text{s.t.} 
\begin{cases}
    Ax \le b \\
    x \ge 0
\end{cases}
$$
将各个矩阵或向量根据基变量和非基变量分为两部分:
令 $c^T = \begin{bmatrix} c_B^T & c_N^T \end{bmatrix}, A = \begin{bmatrix} A_B & A_N \end{bmatrix}, x = \begin{bmatrix} x_B^T & x_N^T \end{bmatrix}^T$,有
$$
\begin{cases}
A_Bx_B + A_Nx_N = b \\ 
z = c_B^Tx_B + c_N^Tx_N
\end{cases}
$$
通过移项就能用 $x_N$ 表示其它变量:
$$
\begin{cases}
x_B = A_B^{-1}b - A_B^{-1}A_Nx_N \\ 
z = c_B^TA_B^{-1}b + (c_N^T - c_B^TA_B^{-1}A_N)x_N
\end{cases}
$$
由于基可行解中 $x_N = 0$,有
$$
\begin{cases}
x_B = A_B^{-1}b \\ 
z = c_B^TA_B^{-1}b
\end{cases}
$$
首先画一张这样的表:
$$
\begin{array}{c|cc|c} & c_B^T & c_N^T & 0 \\ \hline x_B & A_B & A_N & b\end{array}
$$
利用行变换将 $A_B$ 变为 $I$,有
$$
\begin{array}{c|cc|c} & c_B^T & c_N^T & 0 \\ \hline x_B & I & A_B^{-1}A_N & A_B^{-1}b\end{array}
$$
接着把下面一行乘上 $c_B^T$,用上面一行相减,有
$$
\begin{array}{c|cc|c} & 0 & c_N^T-c_B^TA_B^{-1}A_N & -c_B^TA_B^{-1}b \\ \hline x_B & I & A_B^{-1}A_N & A_B^{-1}b\end{array}
$$
表格每计算一次,即为单纯形法的一次迭代。 \\~\\
假设有以下问题:(和上一节例子相同)
$$
\max \quad 3x_1 + 2x_2
$$
$$
\text{s.t.} 
\begin{cases}
    2x_1 + x_2 \le 12 \\
    x_1 + 2x_2 \le 9 \\
    x_1, x_2 \ge 0
\end{cases}
$$
加入松弛变量后,有
$$
\max \quad 3x_1 + 2x_2
$$
$$
\text{s.t.} 
\begin{cases}
    2x_1 + x_2 + x_3 = 12 \\
    x_1 + 2x_2 + x_4 = 9 \\
    x_1, x_2, x_3, x_4 \ge 0
\end{cases}
$$
初始单纯形表:
$$
\begin{array}{c|cccc|c} & 3 & 2 & 0 & 0 & 0 \\ \hline x_3 & 2 & 1 & 1 & 0 & 12 \\ x_4 & 1 & 2 & 0 & 1 & 9 \end{array}
$$
选择检验数中较大的那个($3$,对应 $x_1$)。由于 $12/2 = 6 < 9/1 = 9$,所以 $x_1$ 成为基变量,$x_3$ 被移出基变量。修改表格为
$$
\begin{array}{c|cccc|c} & 0 & 1/2 & -3/2 & 0 & -18 \\ \hline x_1 & 1 & 1/2 & 1/2 & 0 & 6 \\ x_4 & 0 & 3/2 & -1/2 & 1 & 3 \end{array}
$$
选择最大的检验数($1/2$,对应 $x_2$),由于 $3/(3/2) = 2 < 6/(1/2) = 12$,所以 $x_2$ 成为基变量,$x_4$ 被移出基变量。修改表格为
$$
\begin{array}{c|cccc|c} & 0 & 0 & -4/3 & -1/3 & -19 \\ \hline x_1 & 1 & 0 & 2/3 & -1/3 & 5 \\ x_2 & 0 & 1 & -1/3 & 2/3 & 2 \end{array}
$$
此时所有检验数都为非正数,单纯形法结束。单纯形表右上角的 $-z$ 就是最优目标函数的相反数,所以最优目标函数为 $19$。表格最右一列的 $A_B^{-1}b$ 就是 $x_B$ 的取值,所以最优解为 $x_1 = 5$,$x_2 = 2$。

\subsection{退化}
退化 (degeneration) 指一个基可行解中,存在至少一个基变量为 $0$ 的情况,即这个基变量可以和另一个非基变量任意互换,而不影响结果。 \\
可以通过以下规则,避免单纯形法进入循环:\textbf{在有多个可以入基的变量时,选择下标最小的一个;在有多个可以出基的变量时,也选择下标最小的一个}。 \\
接下来证明\textbf{单纯形法在非退化的情况下必定可以取到最优解}: \\~\\
首先证明:若检验数向量 $\hat{c} = c_N - c_B^TA_B^{-1}A_N$ 的每一维都小等于 $0$,那么此时基可行解 $x$ 是最优解。 \\
利用反证法,假设有一个更优的 $y$ 才是最优解。令 $d = x - y$,有 
$$
Ax - Ay = b - b = 0 = Ad = A_Bd_B + A_Nd_N
$$
令 $d_B = -A_B^{-1}A_Nd_N$ 则
$$
c^Td = c_B^Td_B + c_N^Td_N = (c_N^T-c_B^TA_B^{-1}A_N)d_N = \hat{c}d_N
$$
根据基可行解的定义 $x_N = 0$,又 $y \ge 0$,因此 $d_N \ge 0$;又 $\hat{c} \le 0$,所以 $c^Td = \hat{c}d_N \le 0$,则 $c^Ty = c^Tx + c^Td \le c^Tx$,说明 $y$ 并没有比 $x$ 更优,矛盾。这说明了在检验数均小等于 $0$ 时停止算法,就能得到最优解。 \\
类似地,可以说明若基可行解 $x$ 是非退化的最优解,检验数向量的每一维都小等于 $0$。否则由于 $x$ 是非退化的最优解,如果有一个检验数大于 $0$,它对应的非基变量一定有增加的空间,就能构造一个更优的解。这说明了在非退化的情况下,肯定有解满足算法停止的情况。

\subsection{大M法}
大M法 (big M method)将线性规划的目标函数改为
$$
z = c^Tx - M\sum_{i=1}^m\bar{x}_i
$$
如果 $M$ 是一个足够大的正数,如果原问题存在可行解,$\bar{x}$ 就会在 $-M$ 这个“惩罚”之下变成 $0$。
然而, $M$ 的值无从定义。如果 $M$ 的值取得太小导致 $\bar{x}$ 最后仍非 $0$,此时,有两种可能,一是 $M$ 太小了,另一个则是因为问题没有可行解;如果 $M$ 的值取得太大,可能会带来计算上的误差。

\subsection{两阶段法}
两阶段法 (two-phase method) 将原问题改变为只由松弛变量组成的优化问题,解决了这个优化问题,就找到了原问题的一个可行解。 \\
有优化问题如下:
$$
\min \quad \sum_{i = 1}^{m} \bar{x}_i
$$
$$
\text{s.t.} 
\begin{cases}
    Ax + \bar{x} = b \\
    x, \bar{x} \ge 0
\end{cases}
$$
对于这个优化问题,$\bar{x} = b$ 就是一个可行解。 \\
如果这个优化问题的最优解的目标函数值不为 $0$,原问题无可行解;如果最优解让目标函数值为 $0$,则存在一种 $x$ 的取值满足约束,且 $\bar{x} = 0$,这样就找到了原问题的一个可行解。再以这个可行解为起点,利用单纯形法求出原问题的最优解即可。 \\~\\
假设有以下问题:
$$
\max \quad 4x_1 - x_2 + x_3
$$
$$
\text{s.t.} 
\begin{cases}
    x_1 + 2x_2 + 3x_3 = 1 \\
    2x_1 + 3x_2 + 2x_3 = 2 \\
    x \ge 0
\end{cases}
$$
加入松弛变量后,转化为两阶段法的优化问题:
$$
\max \quad -x_4 - x_5
$$
$$
\text{s.t.} 
\begin{cases}
    x_1 + 2x_2 + 3x_3 + x_4 = 1 \\
    2x_1 + 3x_2 + 2x_3 + x_5 = 2 \\
    x \ge 0
\end{cases}
$$
初始单纯形表:
$$
\begin{array}{c|ccccc|c} & 3 & 5 & 5 & 0 & 0 & 3 \\ \hline x_4 & 1 & 2 & 3 & 1 & 0 & 1 \\ x_5 & 2 & 3 & 2 & 0 & 1 & 2\end{array}
$$
选择 $x_1$ 入基,$x_4$ 出基,有
$$
\begin{array}{c|ccccc|c} & 0 & -1 & -4 & -3 & 0 & 0 \\ \hline x_1 & 1 & 2 & 3 & 1 & 0 & 1 \\ x_5 & 0 & -1 & -4 & -2 & 1 & 0 \end{array}
$$
此时,目标函数值已经是 $0$ 了,但是基变量里有一个 $x_5$,这是一个退化情况,有 $x_2 = x_3 = x_5 = 0$。此时可以让 $x_2$ 入基,$x_5$ 出基,就能把基变量变为 $x_1$ 和 $x_2$。同时也求出了原问题的一个可行解:$x_1 = 1, x_2 = x_3 = 0$,基变量是 $x_1$ 和 $x_2$。 \\
利用单纯形表求解原问题,有初始单纯形表:
$$
\begin{array}{c|ccc|c} & 0 & 0 & 25 & -4 \\ \hline x_1 & 1 & 0 & -5 & 1 \\ x_2 & 0 & 1 & 4 & 0 \end{array}
$$
这一个退化情况。让 $x_3$ 入基,$x_2$ 出基,有
$$
\begin{array}{c|ccc|c} & 0 & -25/4 & 0 & -4 \\ \hline x_1 & 1 & 5/4 & 0 & 1 \\ x_3 & 0 & 1/4 & 1 & 0 \end{array}
$$
所有检验数都非正,迭代结束。得到了原问题的最优解:$x_1 = 1, x_2 = x_3 = 0$,此时目标函数值为 $4$。
\pagebreak
