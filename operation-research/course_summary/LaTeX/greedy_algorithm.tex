\section{贪心算法}
\subsection{一类贪心算法}
\subsubsection{一类贪心算法}
考虑一个有限元素集合 $E$,给 $E$ 中的每个元素 $e$ 定义一个非负的费用 $c(e)$。再考虑 $\mathcal{F} \in 2^E$,对于 $F \in \mathcal{F}$,定义 $F$ 的费用 $c(F) = \sum\limits_{e \in F} c(e)$。一类问题即找出一个 $F$,使得 $c(F)$ 最大(或最小)。

\subsubsection{独立集系统}
对于一个二元组 $(E, \mathcal{F})$,
\begin{enumerate}
    \item $\emptyset \in \mathcal{F}$ 。
    \item $\forall Y \in \mathcal{F}$,$X \subseteq Y \to X \in \mathcal{F}$ 。 
\end{enumerate}
称 $(E, \mathcal{F})$ 为独立系统 (independent system)。 \\
在独立系统 $(E, \mathcal{F})$ 中,$\mathcal{F}$ 中的元素称为独立集,$E - \mathcal{F}$ 中的元素称为相关集。

\subsubsection{基和圈}
将 $\mathcal{F}$ 中的极大独立集称为基 (basis),将 $E - \mathcal{F}$ 中的极小相关集称为圈 (circuit)。
对于 $X \subseteq E$,定义 $X$ 上的基为 $X$ 中的极大独立集。

\subsubsection{秩商}
对于 $X \subseteq E$,$X$ 中的基大小可能不同。定义 $X$ 的秩 $r(X)$ 为 $X$ 中最大的基的大小,类似地定义 $X$ 的下秩 $\rho(X)$ 为 $X$ 中最小的基的大小。
由此定义独立系统的秩商 $q(E, \mathcal{F}) = \min\limits_{x \subseteq E} \quad \frac{\rho(X)}{r(X)}$。秩商是一类问题中贪心解法近似比的下界。

\subsubsection{独立集系统的对偶}
$(E, \mathcal{F}) \rightarrow (E, \mathcal{F}^*)$,$\mathcal{F}^* = \{X \subseteq E \ | \ \exists F$ 的一个基 $\beta, \ X \cap \beta = \emptyset\}$。 \\
有以下性质:
\begin{enumerate}
    \item $(E, \mathcal{F}^*)$是一个独立集系统。
    \item 若$B$是$(E, \mathcal{F})$的基,$E - B$是$(E, \mathcal{F}^*)$的基。
    \item $(E, \mathcal{F}^{**}) = (E, \mathcal{F})$。 \\
    证明:
    \begin{align}
        \nonumber
        \forall X \in \mathcal{F}^{**} & \iff 
        \exists (E, \mathcal{F}^*) \mbox{的一个基}B^*, X \cap B^* = \emptyset \\ & \iff \nonumber
        \exists (E, \mathcal{F}) \mbox{的一个基}B, X \cap (E - B) = \emptyset \\ & \iff \nonumber
        \exists (E, \mathcal{F}) \mbox{的一个基}B, X \subseteq B \\ & \iff \nonumber
        X \subseteq \mathcal{F} \nonumber
    \end{align}
\end{enumerate}

\subsection{一类最大(小)化问题}
根据独立系统的定义,可以引出一类最大(小)化问题。
\begin{itemize}
    \item 最大化问题:给出一个独立系统 $(E, \mathcal{F})$,找出一个 $F \in \mathcal{F}$,使得 $c(F)$ 最大。
    由于每个元素的费用都是非负的,所以 $|F|$ 越大,$c(F)$ 也越大,因此最优的 $F$ 一定是基。
    \item 最小化问题:给出一个独立系统 $(E, \mathcal{F})$,找出一个 $F \in \mathcal{F}$,使得 $F$ 是基,且 $c(F)$ 最小。
\end{itemize}
对任一最大化问题$(E, \mathcal{F})$有$q(E, \mathcal{F}) \le \frac{G(E, \mathcal{F})}{OPT(E, \mathcal{F})} \le 1$,其中$G(E, \mathcal{F})$是贪心算法的目标函数值。 \\
以下证明: \\
$E = \{e_1, \ \cdots, \ e_n\}, \ c: E \rightarrow R^+$。$G_n$是贪心算法的解,$O_n$是最优解,有$G_j = G_n \cap E_j, \ O_j = O_n \cap E_j$,其中$E_j = \{e_1, \ \cdots, \ e_j\}, \ c_1 \le c_2 \le \cdots, c_n, \ G_0 = O_0 = \emptyset$。
\begin{align}
    c(G_n) & = \sum^{n}_{j = 1}(|G_j| - |G_{j - 1}|) \times c_j \tag{1} \\ 
    & = \sum^{n}_{j = 1}|G_j| \times (c_j - c_{j + 1}) \tag{2} \\
    & \ge \sum^{n}_{j = 1}\rho(E_j) \times (c_j - c_{j + 1}) \tag{3} \\ 
    & \ge q\sum^{n}_{j = 1}r(E_j)(c_j - c_{j + 1}) \tag{4} \\
    & \ge q\sum^{n}_{j = 1}|O_j|(c_j - c_{j + 1}) \tag{5} \\ 
    & = q \times c(O_n) \tag{6} \\
    & \Rightarrow \quad \frac{c(G_n)}{c(O_n)} \ge q \tag{7}
\end{align}
($(2) \Rightarrow (3)$:因为$|G_j|$是$E_j$的一个最大独立集;$(4) \Rightarrow (5)$:因为$r(E_j)$是最大个数) \\
$\exists F, \ \frac{\rho(F)}{r(F)} = q(E, F) = \frac{|X|}{|Y|}$, 构造
$$
c(e) = 
\begin{cases}
1, \ e \in F \\
0, \ e \notin F
\end{cases} \\
(c_1 = c_2 = \cdots = c_{|X|} = 1, \ c_{|X| + 1} = \cdots = c_n = 0)
$$

\subsubsection{最大化问题的实例}
\begin{itemize}
    \item $0-1$ 背包问题:$E$ 中的元素是每个物品,$\mathcal{F}$ 中的元素是所有可以放进背包的物品集合,费用就是物品的价值。
    \item 最大权独立集:$E$ 中的元素是点,$\mathcal{F}$ 中的元素是独立集,费用就是每个点的权值。
    \item 最长简单路径:$E$ 中的元素是边,$\mathcal{F}$ 中的元素是所有从起点到终点的简单路径以及其子集,费用就是每条边的距离。
    \item 最大权森林:$E$ 中的元素是边,$\mathcal{F}$ 中的元素是所有不含圈的边集,费用就是每条边的权值。
\end{itemize}

\subsubsection{最小化问题的实例}
\begin{itemize}
    \item 最小生成树问题:$E$ 中的元素是边,$\mathcal{F}$ 中的元素是所有不含圈的边集,费用就是每条边的权值。
    \item 最短路问题:$E$ 中的元素是边,$\mathcal{F}$ 中的元素是所有从起点到终点的简单路径以及其子集,费用就是每条边的距离。
    \item 旅行商问题:$E$ 中的元素是边,$\mathcal{F}$ 中的元素是哈密尔顿回路及其子集,费用就是每条边的距离。
\end{itemize}

\subsection{拟阵}
拟阵 (matroid)是一个特殊的独立集系统。一个独立系统需要满足以下三个条件中的一个才被称为是拟阵(以下三个条件等价):
\begin{enumerate}
    \item 若 $X, Y \in \mathcal{F}$,且 $|X| > |Y|$,则 $\exists e \in X - Y$,$Y \cup \{e\} \in \mathcal{F}$。
    \item 若 $X, Y \in \mathcal{F}$,且 $|X| = |Y| + 1$,则 $\exists e \in X - Y$,$Y \cup \{e\} \in \mathcal{F}$。
    \item $\forall X \subseteq E$,$X$ 的所有基大小相同。
\end{enumerate}

\subsubsection{例子}
\begin{enumerate}
    \item $E = \{v_1, \ \cdots, \ v_m\}$,$\mathcal{F} = \{z \subseteq E \ | \ z$是线性无关组$\}$。
    \item $E$是有限集合,$\mathcal{F} = \{X \subseteq E, |X| \le k \in N\}$。(一致拟阵)
    \item $E$是无向图$G$,$\mathcal{F} = \{X \subseteq E, X$构成一个森林$\}$。(图拟阵)
\end{enumerate}

\subsubsection{拟阵的交}
假设有两拟阵$(E, \mathcal{F}_1)$,$(E, \mathcal{F}_2)$,其交为$(E, \mathcal{F}_1 \cap \mathcal{F}_2)$。 \\
$n$ 个拟阵的交,用贪心算法得到的解近似比为 $\frac{1}{n}$。 \\~\\
任一个独立集系统都是有限个拟阵的交。 \\
以下证明: \\
独立集系统$(E, \mathcal{F})$,假设其有$k$个圈$C_1, \ \cdots, C_k$,$F_i = \{X \subseteq E | X \nsupseteq C_i\}$。 \\
$(E, \mathcal{F}_i), \ \forall \mathcal{F} \subseteq E$,此时有两种可能:
\begin{enumerate}
    \item $F \nsupseteq C_i$。
    \item $F \supseteq C_i$,$\forall e \in C_i, \mathcal{F} - \{e\}$是极大独立集。
\end{enumerate}
$\forall X \in \mathcal{F}, X$不含任何圈$\iff X \in \cap^{k}_{i = 1}\mathcal{F}_i$ \\
$(E, \mathcal{F})$为独立集系统,$\mathcal{F} = \cap^{k}_{i = 1}\mathcal{F}_i$。 \\
设$F \subseteq E$,考虑$(E, \mathcal{F})$在$\mathcal{F}$上,两个不同的极大独立集,$A, B \ (|B| \ge |A|)$欲证$|B| \le k|A|$。 \\
$\forall e \in B - A$,有$e \notin \cap^{k}_{i = 1} (A_i - A)$,$A_i: (E, \mathcal{F}_i)$在$A \cup B$含$A$极大独立集,则$e$最多出现在$(k - 1)$个$A_i - A$中,$\sum^{k}_{i = 1}|A_i - A| \le (k - 1)|B - A| \le (k - 1)|B|$,
同理$k|B| \le \sum^{k}_{i = 1}|A_i| $(因为极大独立集有相同维数)$\le k|A| + (k - 1)|B|$。 \\
得证$|B| \le k|A|$。

\subsection{两类贪心算法}
\subsubsection{Best in}
将 $E$ 中所有元素按费用从大到小排序,使得 $c(e_1) \ge c(e_2) \ge ... \ge c(e_n)$。假设 $F = \emptyset$,依照 $e_1, e_2 \dots, e_n$ 的顺序考虑,若 $e_i$ 加入 $F$ 后 $F$ 仍是独立集,就加入$e_i$。此贪心算法用于解决最大化问题。 \\
设 $G(E, \mathcal{F})$ 表示 Best in 算法得到的解;$\text{OPT}(E, \mathcal{F})$ 表示最优解,则 $$q(E, \mathcal{F}) \le \frac{G(E, \mathcal{F})}{\text{OPT}(E, \mathcal{F})} \le 1$$ 从此定理可以看出:若一个独立集系统是拟阵,则用 Best in 算法得到的最大化问题的解一定是最优解。

\subsubsection{Worst out}
将 $E$ 中所有元素按费用从大到小排序,使得 $c(e_1) \ge c(e_2) \ge ... \ge c(e_n)$。假设 $F = E$,依照 $e_1, e_2 \dots, e_n$ 的顺序考虑,若把 $e_i$ 从 $F$ 中去掉后 $F$ 还含有至少一个基,就去掉$e_i$。此贪心算法用于解决最小化问题。 \\
使用 Worst out 贪心得到的解满足 
$$
1 \le \frac{G(E, \mathcal{F})}{\text{OPT}(E, \mathcal{F})} \le \max\limits_{F \subseteq E} \quad \frac{|F| - \rho^*(F)}{|F| - r^*(F)}
$$ 
其中 $\rho^*(F)$ 表示对偶独立集系统中的下秩,$r^*(F)$ 表示对偶独立系统中的秩。

\pagebreak
